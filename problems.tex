\documentclass{article}

\usepackage{amsmath, amsfonts, geometry}

\begin{document}

\author{Tanner Prynn}
\title{Math 415B Project - Worked Problems}
\maketitle

%\section*{Discussion}
%I picked a selection of 10 problems from Silverman and Washington which I felt were the most relevant problems to my project. Washington provides a section on cryptography but only a single question about Diffie-Hellman (with many questions about outdated El-Gamal), which left me with a lack of questions related to cryptographic applications. Hopefully the coding aspect substitutes for that to some extent.

%Some of these questions are a bit long, so I'm not sure if I should complete them all (thoughts?).

%My thoughts on the order of completing the problems (grouped by most to least relevant):
%\begin{itemize}
%\item Silverman x1.2, x1.11, x1.12, x1.19
%\item Silverman x1.17, x1.18; Washington x2.5
%\item Silverman o1.14; Washington x2.17, o6.1
%\end{itemize}

\section{Silverman 1.2}
Let $C$ be the conic given by the equation
$$F(x,y) = ax^2 + bxy + cy^2 + dx + ey + f = 0$$
and let $\delta$ be the determinant
$$\det \begin{bmatrix}
2a & b & d \\
b & 2c & e \\
d & e & 2f \\
\end{bmatrix}$$

(a) Show that if $\delta \neq 0$ then $C$ has no singular points. That is, show there are no points $(x,y)$ where
$$F(x,y) = \frac{\partial F}{\partial x}(x,y) = \frac{\partial F}{\partial y}(x,y) = 0$$

In homogenous form we have:
$$\bar F = ax^2 + bxy + cy^2 + dxz + eyz + fz^2$$
$$\frac{\partial \bar F}{\partial x} = 2ax + by + dz$$
$$\frac{\partial \bar F}{\partial y} = bx + 2cy + ez$$
$$\frac{\partial \bar F}{\partial z} = dx + ey + 2fz$$

$\delta \neq 0$ implies that the matrix is invertible, i.e. that the only solution to $Ax = 0$ is $x = 0$. But there is no point $(0,0,0)$ in projective space. So $C$ has no singular points.

(b) Conversely, show that if $\delta = 0$ and $b^2 - 4ac \neq 0$ then there is a unique singular point on $C$.

If we have a degenerate conic ($\delta = 0$) then we can classify it on whether $J = b^2 - 4ac$ is greater, equal, or less than zero. If $b^2-4ac < 0$, then the conic is two intersecting lines, and has one singular point (the point of intersection). If $b^2 - 4ac > 0$ then the conic is a single point. Thus if $b^2-4ac \neq 0$ then there is a unique singular point on $C$.

(c) Let $L$ be the line $y = \alpha x + \beta$ with $\alpha \neq 0$. Show that the intersection of $L$ and $C$ consists of either zero, one, or two points.

Substitute the equation of the line into the equation for the conic.

\begin{align*}
0 &= ax^2 + bx(\alpha x + \beta) + c(\alpha x + \beta)^2 + dx + e(\alpha x + \beta) + f \\
&= ax^2 + \alpha b x^2 + \beta b x + \alpha^2 c x^2 + 2\alpha\beta c x + c\beta^2 + dx + \alpha e x + \beta e + f \\
&= (a + \alpha b + \alpha^2 c)x^2 + (\beta b + 2\alpha\beta c + d + \alpha e)x + (c\beta^2 + \beta e + f) \\
&= a'x^2 + b'x + c'
\end{align*}

We'll consider this equation in $\mathbb{R}$. It's a quadratic equation which has zero, one, or two solutions depending on the determinant ${b'}^2 - 4a'c'$. If the determinant is zero, there is one solution. If the determinant is positive, there are two solutions. If the determinant is negative, there are no real solutions.

(d) Determine the conditions on the coefficients which ensure that the intersection $L \cap C$ consists of exactly one point. What is the geometric significance of these conditions?

There is exactly one solution if ${b'}^2 - 4a'c' = 0$. In geometric terms, this line must be tangent to the curve.

\section{Silverman 1.11}
Let $S$ be a set with a composition law $*$ satisfying the following two properties:

(i) $P*Q = Q*P$ for all $P,Q \in S$

(ii) $P*(P*Q) = Q$ for all $P,Q \in S$

Fix an element $\mathcal{O} \in S$, and define a new composition law $+$ by the rule $P+Q = \mathcal{O} * (P * Q)$

(a) Prove that $+$ is commutative and has $\mathcal{O}$ as identity element.

Let $P,Q \in S$. Then $P+Q = \mathcal{O} * (P * Q) = \mathcal{O} * (Q * P) = Q+P$ so $+$ is commutative.

$P + \mathcal{O} = \mathcal{O} * (P * \mathcal{O}) = \mathcal{O} * (\mathcal{O} * P) = P$ so $\mathcal{O}$ is the identity.

(b) Prove that for any given $P,Q \in S$, the equation $X+P = Q$ has the unique solution $X = P*(Q*\mathcal{O})$ in $S$. In particular, if we define $-P$ to be $P*(\mathcal{O} * \mathcal{O})$ then $-P$ is the unique solution in $S$ of the equation $X+P = \mathcal{O}$.

Let $J, K, L \in S$. We have cancellation for $*$ as follows:

\begin{align*}
J * K &= J * L\\
J * (J * K) &= J * (J * L)\\
K &= L
\end{align*}

and cancellation works on both sides because $*$ commutes. Then

\begin{align*}
X+P & = Q \\
& = \mathcal{O} * (\mathcal{O} * Q) \\
\mathcal{O} * (X * P) &= \mathcal{O} * (\mathcal{O} * Q) \\
X * P &= \mathcal{O} * Q \\
&= Q * \mathcal{O} \\
&= P * (P * (Q * \mathcal{O})) \\
&= (P * (Q * \mathcal{O})) * P \\
X &= (P * (Q * \mathcal{O}))
\end{align*}

Finally

\begin{align*}
X+P &= \mathcal{O}\\
X &= (P * (\mathcal{O} * \mathcal{O}))\\
X &= -P
\end{align*}

(c) Prove that $+$ is associative (and so $(S,+)$ is a group) if and only if:

(iii) $R*(\mathcal{O}*(P*Q)) = P*(\mathcal{O}*(Q*R))$ for all $P,Q,R \in S$

($\Leftarrow$) Let $P,Q,R \in S$ and assume (iii).

\begin{align*}
P + (Q + R) &= P + (\mathcal{O} * (Q * R)) \\
&= \mathcal{O} * (P * (\mathcal{O} * (Q * R))) \\
&= \mathcal{O} * (R * (\mathcal{O} * (P * Q))) \\
&= R + (\mathcal{O} * (P * Q)) \\
&= R + (P + Q) \\
&= (P + Q) + R
\end{align*}

($\Rightarrow$) Let $P,Q,R \in S$ and assume $+$ is commutative.

\begin{align*}
P + (Q + R) &= (P + Q) + R \\
&= R + (P + Q) \\
\mathcal{O} * (P * (\mathcal{O} * (Q * R))) &= \mathcal{O} * ( R * (\mathcal{O} * (P * Q))) \\
P * (\mathcal{O} * (Q * R)) &= R * (\mathcal{O} * (P * Q))
\end{align*}

(d) Let $\mathcal{O}' \in S$ be another point, and define a composition law $+'$ by $P +' Q = \mathcal{O}' * (P*Q)$. Suppose that both $+$ and $+'$ are associative, so we obtain two group structures $(S,+)$ and $(S,+')$ on $S$. Prove that the map $\phi: P \mapsto \mathcal{O}*(\mathcal{O}'*P)$ is a group isomorphism from $(S,+)$ to $(S,+')$.

\begin{align*}
\phi(P+Q) &= \mathcal{O} * (\mathcal{O}' * (P + Q)) \\
&= \mathcal{O}' + (P+Q) \\
&= (\mathcal{O}'+P)+Q \\
&= ((\mathcal{O}' +' \mathcal{O}')+P)+Q \\
&= (\mathcal{O}' +' (\mathcal{O}'+P))+Q \\
&= ((\mathcal{O}'+P) +' \mathcal{O}')) + Q \\
&= (\mathcal{O}'+P) +' (\mathcal{O}' + Q) \\
&= \phi(P) +' \phi(Q)
\end{align*}
So $\phi$ is a group homomorphism from $(S,+)$ to $(S,+')$.

Let $\phi(P) = \mathcal{O}'$ for some $P \in S$.
\begin{align*}
\mathcal{O}' &= \phi(P) \\
\mathcal{O}' &= \mathcal{O}' + P \\
\mathcal{O}' + \mathcal{O} &= \mathcal{O}' + P \\
\mathcal{O} &= P
\end{align*}
So $\ker\phi = \{\mathcal{O}\}$, and thus $\phi$ is an isomorphism.

%(e) Find a set $S$ with a composition law $*$ satisfying (i) and (ii) such that $(S,+)$ is not a group.

\section{Silverman 1.12}
The cubic curve $u^3 + v^3 = \alpha$ (with $\alpha \neq 0$) has a rational point $[1,-1,0]$ at infinity. Taking this rational point to be $\mathcal{O}$, we can make the points on the curve into a group.

(a) Derive a formula for the sum $P_1 + P_2$ of two points $P_1 = (u_1, v_1)$ and $P_2 = (u_2, v_2)$.

Line between $P_1$ and $P_2$:
$$m = \frac{v_2 - v_1}{u_2 - u_1}$$
$$b = v_1 - mu_1$$
$$v = mu + b$$

Substitute:
$$u^3 + (mu + b)^3 = \alpha$$
$$u^3 + b^3 + 3b^2mu + 3bm^2u^2 + m^3u^3 - \alpha = 0$$
$$(1+m^3) u^3 + 3bm^2 u^2 + 3b^2m u + b^3 - \alpha = 0$$
$$u^3 + \frac{3bm^2}{1+m^3}u^2 + \frac{3b^2m}{1+m^3}u + \frac{b^3-\alpha}{1+m^3} = 0$$

We know two roots:
$$(u-u_1)(u-u_2)(u-u_3) = 0$$

Coefficient of $u^2$ is $-u_1-u_2-u_3$, so:
$$u_1 + u_2 + u_3 = -\frac{3bm^2}{1+m^3}$$

Thus we have solved for the third point $(u_3,v_3)$:
$$u_3 = - u_1 - u_2 - \frac{3bm^2}{1+m^3}$$
$$v_3 = mu_3 + b$$

The third point $P_3 = P_1 + P_2$ then has the coordinates $P_3 = (u_3, -v_3)$.

(b) Derive a duplication formula for $2P$ in terms of $P = (u,v)$.

We have by implicit differentation that:
$$3u^2 - 3v^2 \frac{dv}{du} = 0$$
$$m = \frac{dv}{du} = -\frac{u^2}{v^2}$$

Then we can use the above formula with this slope to find the new point.

%\section{Silverman 1.14}
%Let $C$ be the cubic curve $u^3 + v^3 = u+v+1$. In the projective plane this curve has a point $[1,-1,0]$ at infinity. Find rational functions $x = x(u,v)$ and $y=y(u,v)$ so that $x$ and $y$ satisfy a cubic equation in Weierstrass normal form with the given point still at infinity.

\section{Silverman 1.17}
Let $C$ be a cubic curve in the projective plane given by the homogenous equation 
$$Y^2 Z = X^3 + aX^2 Z + bXZ^2 + cZ^3$$
Verify that the point $[0,1,0]$ at infinity is a non-singular point of $C$.

Let $\bar F = X^3 + aX^2 Z + bXZ^2 + cZ^3 - Y^2Z$. We have the following partial derivatives:
\begin{align*}
\frac{\partial \bar F}{\partial x} &= 3Z^2 + 2aZX + bZ^2 \\
\frac{\partial \bar F}{\partial y} &= 2ZY \\
\frac{\partial \bar F}{\partial z} &= 3cZ^2 + 2bXZ + aX^2 - Y^2
\end{align*}

Evaluating at the point $[0,1,0]$ we find that
\begin{align*}
\frac{\partial \bar F}{\partial x} &= 0 \\
\frac{\partial \bar F}{\partial y} &= 0 \\
\frac{\partial \bar F}{\partial z} &= -1
\end{align*}

So $\frac{\partial \bar F}{\partial z} \neq 0$ and thus $[0,1,0]$ is non-singular.

\section{Silverman 1.18}
The cubic curve $y^2 = x^3 + 17$ has the following five rational points:
$$P_1 = (-2,3), P_2 = (-1,4), P_3 = (2,5), P_4 = (4,9), P_5 = (8,23)$$

(a) Show that $P_2$, $P_4$, and $P_5$ can each be expressed as $mP_1 + nP_3$ for an appropriate choice of integers $m$ and $n$.

Initially I wrote code to solve this problem. Then I realized that I needed to represent rational numbers exactly, and decided it would be simpler to do something else.

Using the Pari/GP software I found the following:
\begin{verbatim}
? e1=ellinit([0,0,0,0,17])
? {
for(m = -5, 5,
  for(n = -5, 5,
    print(m); print(n);
    print(elladd(e1, ellmul(e1, [-2,3], m), ellmul(e1, [2,5], n)))
  )
)
}
\end{verbatim}

\begin{align*}
P_2 &= -2P_1 + P_2 \\
P_4 &= P_1 + -P_2 \\
P_5 &= -2P_1 + 0P_2
\end{align*}

(b) Compute the points $P_6 = -P_1 + 2P_3$ and $P_7 = 3P_1 - P_3$.
\begin{verbatim}
? elladd(e1, ellmul(e1, [-2,3], -1), ellmul(e1, [2,5], 2))
%6 = [43, 282]
? elladd(e1, ellmul(e1, [-2,3], 3), ellmul(e1, [2,5], -1))
%7 = [52, 375]
\end{verbatim}

(c) Notice that the points $P_1$ to $P_7$ all have integer coordinates. There is exactly one more rational point on this curve which has integer coordinates and $y > 0$. Find that point.

From (a), I have the point $-4P_1 + 3P_3 = (5234, 378661)$.

\section{Silverman 1.19}
Suppose that $P = (x,y)$ is a point on the cubic curve
$$y^2 = x^3 + ax^2 + bx + c$$

(a) Verify that the $x$ coordinate of the point $2P$ is given by the duplication formula
$$x(2P) = \frac{x^4 - 2bx^2 - 8cx - 4ac + b^2}{4y^2}$$

\begin{align*}
x(2P) &= \lambda^2 - a - 2x \\
\lambda &= \frac{3x^2 + 2ax + b}{2y} \\
x(2P) &= (\frac{3x^2 + 2ax + b}{2y})^2 - a - 2x \\
x(2P) &= \frac{9x^4 + 12ax^3 + (4a^2 + 6b)x^2 + 4abx + b^2}{4y^2} - a - 2x \\
x(2P) &= \frac{9x^4 + 12ax^3 + (4a^2 + 6b)x^2 + 4abx + b^2}{4y^2} - a\frac{4y^2}{4y^2} - 2x\frac{4y^2}{4y^2} \\
x(2P) &= \frac{9x^4 + 12ax^3 + (4a^2 + 6b)x^2 + 4abx + b^2 - 4ay^2 - 8xy^2}{4y^2}
\end{align*}

\begin{align*}
4ay^2 &= 4a(x^3 + ax^2 + bx + c) \\
&= 4ax^3 + 4a^2x^2 + 4abx + 4ac \\
8xy^2 &= 8x(x^3 + ax^2 + bx + c) \\
&= 8x^4 + 8ax^3 + 8bx^2 + 8cx
\end{align*}

\begin{align*}
& 9x^4 + 12ax^3 + (4a^2 + 6b)x^2 + 4abx + b^2 - 4ay^2 - 8xy^2\\
=~& 9x^4 + 12ax^3 + (4a^2 + 6b)x^2 + 4abx + b^2 - (4ax^3 + 4a^2x^2 + 4abx + 4ac) - (8x^4 + 8ax^3 + 8bx^2 + 8cx) \\
=~& (9x^4 - 8x^4) + (12ax^3 - 4ax^3 - 8ax^3) + (4a^2x^2 + 6bx^2 - 4a^2x^2 - 8bx^2) + (4abx - 4abx - 8cx) + (b^2 - 4ac) \\
=~& x^4 - 2bx^2 - 8cx + b^2 - 4ac
\end{align*}

Thus
$$x(2P) = \frac{x^4 - 2bx^2 - 8cx - 4ac + b^2}{4y^2}$$

(b) Derive the formula for the $y$ coordinate of $2P$ in terms of $x$ and $y$.
$$v = y - \lambda x$$
\begin{align*}
y(2P) &= \lambda x(2P) + v \\
&= \frac{3x^2 + 2ax + b}{2y}\frac{x^4 - 2bx^2 - 8cx - 4ac + b^2}{4y^2} + y - \lambda x \\
&= \frac{(3x^2 + 2ax + b)(x^4 - 2bx^2 - 8cx - 4ac + b^2)}{8y^3} + y - \frac{3x^3 + 2ax^2 + bx}{2y}
\end{align*}

(c) Find a polynomial in $x$ whose roots are the $x$ coordinates of the points $P=(x,y)$ satisfying $3P=\mathcal{O}$. (Hint: the relation $3P = \mathcal{O}$ can be written $2P = -P$.)

\begin{align*}
\frac{x^4 - 2bx^2 - 8cx - 4ac + b^2}{4y^2} &= x \\
x^4 - 2bx^2 - 8cx - 4ac + b^2 &= 4xy^2 \\
&= 4x(x^3 + ax^2 + bx + c) \\
&= 4x^4 + 4ax^3 + 4bx^2 + 4cx \\
0 &= 4x^4 - x^4 + 4ax^3 + 4bx^2 + 2bx^2 + 4cx + 8cx + 8ac - b^2 \\
0 &= 3x^4 + 4ax^3 + 6bx^2 + 12cx + 8ac - b^2
\end{align*}

(d) For the curve $y^2 = x^3 + 1$, solve the equation in (c) to find all of the points satisfying $3P = \mathcal{O}$.

We have $a = 0$, $b = 0$, and $c = 1$. We can thus simplify our equation:
\begin{align*}
0 &= 3x^4 + 4ax^3 + 6bx^2 + 12cx + 8ac - b^2 \\
&= 3x^4 + 12x \\
&= 3x(x^3 + 4)
\end{align*}

Which has zeroes at $x = 0,-2^\frac{2}{3},-i(2^\frac{2}{3}),-(i^\frac{1}{3})(2^\frac{2}{3})$. These correspond to the points:
$$(0, 0),~\left(-2^\frac{2}{3}, \pm \sqrt{-2^\frac{2}{3} + 1}\right),$$
$$\left(-i(2^\frac{2}{3}), \pm \sqrt{-i(2^\frac{2}{3}) + 1}\right),~\left(-(i^\frac{1}{3})(2^\frac{2}{3}), \pm \sqrt{-(i^\frac{1}{3})(2^\frac{2}{3}) + 1}\right)$$

\section{Washington 2.5}
Let $(x,y)$ be a point on the elliptic curve $E$ given by $y^2 = x^3 + Ax + B$. Show that if $y = 0$ then $3x^2 + A \neq 0$. 

If $y = 0$ then $(x,y)$ on $E$ satisfies the following condition:
$$0 = x^3 + Ax + B$$

However, the point must be nonsingular. $\frac{\partial F}{\partial y} = 0$, so $\frac{\partial F}{\partial x}$ must be nonzero, or the point is singular. Thus
$$\frac{\partial F}{\partial x} = 3x^2 + A \neq 0$$

\section{Washington 2.17}
For a curve in Weierstrass form and points $P_1 = (x_1,y_1), P_2 = (x_2,y_2)$, the doubling and addition formulas follow:

$$P_1+P_1 = \left(
	\frac{(3x_1^2+a)^2}{(2y_1)^2}-2x_1,
	\frac{3x_1(3x_1^2+a)}{2y_1}-\frac{(3x_1^2+a)^3}{(2y_1)^3}-y_1
\right)$$

$$P_1+P_2 = \left(
	\frac{(y_2-y_1)^2}{(x_2-x_1)^2}-x_1-x_2,
	\frac{(2x_1+x_2)(y_2-y_1)}{(x_2-x_1)}-\frac{(y_2-y_1)^3}{(x_2-x_1)^3}-y_1
\right)$$

\vskip1ex

(a) Show that $(x,y) \mapsto (x, -y)$ is a group homomorphism from E to itself, for any elliptic curve in Weierstrass form.

Let $\phi((x,y)) = (x,-y)$. Let $P_1 = (x_1,y_1)$ and $P_2 = (x_2,y_2)$.

\begin{align*}
\phi(P_1)+\phi(P_1) &= (x_1,-y_1) + (x_1,-y_1) \\
&= \left(
	\frac{(3x_1^2+a)^2}{(-2y_1)^2}-2x_1,
	\frac{3x_1(3x_1^2+a)}{-2y_1}-\frac{(3x_1^2+a)^3}{(-2y_1)^3}+y_1
\right) \\
&= \left(
	\frac{(3x_1^2+a)^2}{(2y_1)^2}-2x_1,
	\frac{(3x_1^2+a)^3}{(2y_1)^3} - \frac{3x_1(3x_1^2+a)}{2y_1} + y_1
\right) \\
&= \phi \left(
	\left(
		\frac{(3x_1^2+a)^2}{(2y_1)^2}-2x_1,
		\frac{3x_1(3x_1^2+a)}{2y_1}-\frac{(3x_1^2+a)^3}{(2y_1)^3}-y_1
	\right)
\right) \\
&= \phi(P_1+P_1)
\end{align*}

Thus $\phi(P_1 + P_1) = \phi(P_1) + \phi(P_1)$.

\begin{align*}
\phi(P_1) + \phi(P_2) &= (x_1, -y_1) + (x_2, -y_2) \\
&=
\left(
	\frac{(-y_2+y_1)^2}{(x_2-x_1)^2}-x_1-x_2,
	\frac{(2x_1+x_2)(-y_2+y_1)}{(x_2-x_1)}-\frac{(-y_2+y_1)^3}{(x_2-x_1)^3}+y_1
\right)
\\
&= \left(
	\frac{(y_2-y_1)^2}{(x_2-x_1)^2}-x_1-x_2,
	\frac{(y_2-y_1)^3}{(x_2-x_1)^3} - \frac{(2x_1+x_2)(y_2-y_1)}{(x_2-x_1)} - y_1
\right) \\
&= \phi \left(
	\left(
		\frac{(y_2-y_1)^2}{(x_2-x_1)^2}-x_1-x_2,
		\frac{(2x_1+x_2)(y_2-y_1)}{(x_2-x_1)}-\frac{(y_2-y_1)^3}{(x_2-x_1)^3}-y_1
	\right)
\right) \\
&= \phi(P_1+P_2)
\end{align*}

Thus $\phi(P_1 + P_2) = \phi(P_1) + \phi(P_2)$. We ignore the cases with $\infty$ because they are trivial. Therefore $\phi$ is a homomorphism.

\vskip1ex

(b) Show that $(x,y) \mapsto (\zeta x, -y)$, where $\zeta$ is a nontrivial cube root of 1, is an automorphism of the elliptic curve $y^2 = x^3 + B$.

Let $\psi((x,y)) = (\zeta x, -y)$. Define $P_1$ and $P_2$ as in (a).

\begin{align*}
\psi(P_1)+\psi(P_1) &= (\zeta x_1,-y_1) + (\zeta x_1,-y_1) \\
&= \left(
	\frac{(3(\zeta x_1)^2)^2}{(-2y_1)^2}-2\zeta x_1,
	\frac{3\zeta x_1(3(\zeta x_1)^2)}{-2y_1}-\frac{(3 (\zeta x_1)^2)^3}{(-2y_1)^3}+y_1
\right) \\
&= \left(
	\zeta \left( \frac{(3x_1^2)^2}{(2y_1)^2}-2x_1 \right),
	\frac{(3x_1^2)^3}{(2y_1)^3} - \frac{3x_1(3x_1^2)}{2y_1} + y_1
\right) \\
&= \psi \left(
	\left(
		\frac{(3x_1^2)^2}{(2y_1)^2}-2x_1,
		\frac{3x_1(3x_1^2)}{2y_1}-\frac{(3x_1^2)^3}{(2y_1)^3}-y_1
	\right)
\right) \\
&= \psi(P_1+P_1)
\end{align*}

Thus $\psi(P_1+P_1) = \psi(P_1) + \psi(P_1)$.

\begin{align*}
\psi(P_1) + \psi(P_2) &= (\zeta x_1, -y_1) + (\zeta x_2, -y_2) \\
&= \left(
	\frac{(y_2-y_1)^2}{(\zeta x_2-\zeta x_1)^2}-\zeta x_1-\zeta x_2,
	\frac{(y_2-y_1)^3}{(\zeta x_2-\zeta x_1)^3} - \frac{(2\zeta x_1+\zeta x_2)(y_2-y_1)}{(\zeta x_2-\zeta x_1)} + y_1
\right)
\\
&=
\left(
	\frac{(y_2-y_1)^2}{\zeta^2(x_2-x_1)^2}-\zeta x_1-\zeta x_2,
	\frac{(y_2-y_1)^3}{\zeta^3(x_2-x_1)^3} - \frac{\zeta(2x_1+x_2)(y_2-y_1)}{\zeta(x_2-x_1)} + y_1
\right) \\
&=
\left(
	\zeta \left( \frac{(y_2-y_1)^2}{(x_2-x_1)^2}-x_1-x_2 \right),
	\frac{(y_2-y_1)^3}{(x_2-x_1)^3} - \frac{(2x_1+x_2)(y_2-y_1)}{(x_2-x_1)} + y_1
\right) \\
&= \psi \left(
	\left(
		\frac{(y_2-y_1)^2}{(x_2-x_1)^2}-x_1-x_2,
		\frac{(2x_1+x_2)(y_2-y_1)}{(x_2-x_1)}-\frac{(y_2-y_1)^3}{(x_2-x_1)^3}-y_1
	\right)
\right) \\
&= \psi(P_1+P_2)
\end{align*}

Thus $\psi(P_1 + P_2) = \psi(P_1) + \psi(P_2)$. Again, we skip $\infty$. Therefore $\psi$ is a homomorphism.

Let $\psi((X,Y,Z)) = \infty = (0,1,0)$. Then $\zeta X = 0$. $\zeta \neq 0$, so $X = 0$. $-Y = 1$, and $Z = 0$. So $(X,Y,Z) = (0,-1,0)$ which is in the equivalence class of $\infty$. Thus $\ker \psi = \{\infty\}$ and $\psi$ is an automorphism.

\vskip1ex

(c) Show that $(x, y) \mapsto (-x, iy)$, is an automorphism of the elliptic curve $y^2 = x^3 + Ax$.

Let $\tau: (x,y) \mapsto (-x,iy)$.

\begin{align*}
\tau(P_1)+\tau(P_1) &=
(-x_1,iy_1) + (-x_1,iy_1) \\
&= \left(
	\frac{(3(-x_1)^2)^2}{(2iy_1)^2}-2(-x_1),
	\frac{3(-x_1)(3(-x_1)^2)}{2iy_1}-\frac{(3(-x_1)^2)^3}{(2iy_1)^3}-iy_1
\right) \\
&= \left(
	\frac{(3x_1^2)^2}{-(2y_1)^2}+2x_1,
	\frac{-3x_1(3x_1^2)}{2iy_1}-\frac{(3x_1^2)^3}{-i(2y_1)^3}-iy_1
\right) \\
&= \left(
	\frac{(3x_1^2)^2}{-(2y_1)^2}+2x_1,
	i\frac{3x_1(3x_1^2)}{2y_1}-i\frac{(3x_1^2)^3}{(2y_1)^3}-iy_1
\right) \\
&= 
\left(
	\frac{(3x_1^2)^2}{-(2y_1)^2}+2x_1,
	i \left( \frac{3x_1(3x_1^2)}{2y_1}-\frac{(3x_1^2)^3}{(2y_1)^3}-y_1 \right)
\right) \\
&= \tau \left(
	\left(
		\frac{(3x_1^2)^2}{(2y_1)^2}-2x_1,
		\frac{3x_1(3x_1^2)}{2y_1}-\frac{(3x_1^2)^3}{(2y_1)^3}-y_1
	\right)
\right) \\
&= \tau(P_1+P_1)
\end{align*}

Thus $\tau(P_1+P_1) = \tau(P_1)+\tau(P_1)$.

\begin{align*}
\tau(P_1) + \tau(P_2) &= (-x_1, iy_1) + (-x_2, iy_2) \\
&= \left(
	\frac{(iy_2-iy_1)^2}{(-x_2+x_1)^2}+x_1+x_2,
	\frac{(-2x_1-x_2)(iy_2-iy_1)}{(-x_2+x_1)}-\frac{(iy_2-iy_1)^3}{(-x_2+x_1)^3}-iy_1
\right) \\
&= \left(
	\frac{-(y_2-y_1)^2}{(x_2-x_1)^2}+x_1+x_2,
	\frac{-i(2x_1x_2)(y_2-y_1)}{-(x_2-x_1)}-\frac{-i(y_2-y_1)^3}{-1(x_2-x_1)^3}-iy_1
\right) \\
&= \left(
	-\frac{(y_2-y_1)^2}{(x_2-x_1)^2}+x_1+x_2,
	i\left( \frac{(2x_1+x_2)(y_2-y_1)}{(x_2-x_1)}-\frac{(y_2-y_1)^3}{(x_2-x_1)^3}-y_1 \right)	
\right) \\
&= \tau \left(
	\left(
		\frac{(y_2-y_1)^2}{(x_2-x_1)^2}-x_1-x_2,
		\frac{(2x_1+x_2)(y_2-y_1)}{(x_2-x_1)}-\frac{(y_2-y_1)^3}{(x_2-x_1)^3}-y_1
	\right)
\right) \\
&= \tau(P_1+P_2)
\end{align*}

Thus $\tau(P_1+P_2) = \tau(P_1)+\tau(P_2)$.
Therefore $\tau$ is a homomorphism.

Let $\tau((X,Y,Z)) = \infty = (0,1,0)$.
Then $X = -0 = 0$, $iY = 1$, and $Z = 0$. $iY = 1$, so $y = \frac{1}{i} = -i$.
So $(X,Y,Z) = (0, -i, 0)$ which is in the equivalence class of $\infty$.
Thus $\ker \tau = \{\infty\}$ and $\tau$ is an automorphism.

%\section{Washington 6.1}
%Let E be the curve $y^2 = x^3+1$ over $F_q$, where $q \equiv 2 \mod 3$. By proposition 4.33, $E$ is supersingular. Let $\omega \in F_{q^2}$ be a primitive third root of unity. Note that $\omega \not \in F_q$ since the order of $F_q^\times$ is $q-1$, which is not a multiple of $3$.

%Show that the map
%$$\beta: E(\bar F_q) \to E(\bar F_q), ~ (x,y) \mapsto (\omega x, y), ~ \beta(\infty) = \infty$$
%is an isomorphism.

\end{document}